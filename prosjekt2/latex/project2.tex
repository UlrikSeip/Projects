\documentclass[a4paper]{article}
% Import some useful packages
\usepackage[margin=0.5in]{geometry} % narrow margins
\usepackage[utf8]{inputenc}
\usepackage[english]{babel}
\usepackage{hyperref}
\usepackage{listings}
\usepackage{amsmath,graphicx,varioref,verbatim,amsfonts,geometry,amssymb,dsfont,blindtext}
%\usepackage{minted}
\usepackage{amsmath}
\usepackage{xcolor}
\hypersetup{colorlinks=true}
\definecolor{LightGray}{gray}{0.95}
\definecolor{dkgreen}{rgb}{0,0.6,0}
\definecolor{gray}{rgb}{0.5,0.5,0.5}
\definecolor{mauve}{rgb}{0.58,0,0.82}
\definecolor{mygray}{rgb}{0.9,0.9,0.9}
\definecolor{LightGray}{gray}{0.95}
\lstset{frame=tb,
	language=Python,
	aboveskip=3mm,
	belowskip=3mm,
	showstringspaces=false,
	columns=flexible,
	basicstyle={\small\ttfamily},
	numbers=none,
	numberstyle=\tiny\color{gray},
	keywordstyle=\color{blue},
	commentstyle=\color{dkgreen},
	stringstyle=\color{mauve},
	backgroundcolor=\color{mygray}
	%breaklines=true,
	%breakatwhitespace=true,
	%tabsize=3
}
\title{Project 2 in FYS3150}
\author{Bendik Steinsvåg Dalen, Ulrik Seip}
%\renewcommand\thesection.\alph{section}
%\renewcommand\thesection{\Alph{section}}
\renewcommand\thesubsection{\thesection.\alph{subsection}}
\renewcommand\thesubsubsection{\thesubsection.\roman{subsubsection}}
\begin{document}
\maketitle

\section{ABSTRACT}

In this project we have implemented Jacobian algorithm, to find eigenvectors, and their corresponding eigenvalues in tridiagonal matrices. We then used this to model a harmonic oscillator problem in three dimensions, with one and two electrons. This turned out to be a computationally heavy, but relatively accurate method.

\section{INTRODUCTION}

Finding eigenvectors analytically is complicated, can be tedious, and can sometimes be impossible, and this is why it is much more convenient to do so numerically. A common way of doing this is by the application of the Jacobian method. Essentially we rotate one matrix element at a time, always taking the one with the highest absolute value, until all but the diagonal elements are essentially zero. All transformations are also applied to an identity matrix that then turns into our eigenvectors. 

The modelling of an electron in a harmonic oscillator potential is one of the most classical problems in quantum mechanics. This has a simple analytical solution, however the interaction between multiple electrons can be harder to solve analyticaly, but does follow similar principles. Therefore any good numerical solution of the first problem, should very easily be able to calculate the interaction between two or more electrons. What we have done in this experiment is to customize the method described above to solve for one electron, and then used that same method on two electrons. 

\section{METHOD}

\subsection{Implementing the Jacobian algorithm}
The implementation follows a standard recipe:
We start with the relation
\begin{equation*}\cot 2\theta=\tau = \frac{a_{ll}-a_{kk}}{2a_{kl}}.
\end{equation*}
This can be used to find the angle $\theta$ that makes the  non-diagonal matrix elements of the transformed matrix 
$a_{kl} = 0$. The quadratic equation is obtained using $\cot 2\theta=1/2(\cot\theta-\tan\theta)$.

\begin{align}
t^2+2\tau t-1=0
\end{align}
Which gives us
\begin{align}
t = -\tau \pm \sqrt{1+\tau^2}
\end{align}
$c$ and $s$ are then obtained by
\begin{align}
c = \frac{1}{\sqrt{1+t^2}}
\end{align}
and
\begin{align}
s=tc
\end{align}  

We then use the rotational factors $c$ and $s$ to rotate every other element in the matrix according to their position, giving us a new diagonal, that is slightly closer to the eigenvalues, and new matrix elements elsewhere, slightly closer to 0. The same is done for a matrix that tarts out as an identity matrix, with the purpose of transforming it into the original matrix' eigenvectors. This is all done in the programming language Julia for high efficiency. See the documentation in section \ref{rotator.jl} for further explanation.

\subsection{Testing the code}

For testing the algorithm we have implemented two tests. One for checking if the largest element in the matrix is correctly located, and one for testing if the resulting eigenvalues are correct. The first one is more useful for development purposes, whilst the second one is essential for validating that our implementation works correctly. See section \ref{opp_c.jl} for the test functions.

\subsection{Quantum dots in three dimensions, one electron}

Now that we had a general algorithm we used it to model a electron that moves in a three-dimensional harmonic oscillator potential. In other words, we looked for the solution of the radial part of Schroedinger’s
equation for one electron, which reads
\begin{align}
- \frac{\hbar^2}{2m} \left( \frac{1}{r^2} \frac{d}{dr} r^2 \frac{d}{dr} - \frac{l(l+1)}{r^2}\right) R(r) + V(r) R(r) = ER(r).
\end{align}
This problem also has analytical solutions, so we can test how accurate our algorithm is.

We decided to limit our experiment to the ground state of $l=0$. After introducing the constant $\alpha = \left(\frac{\hbar^2}{mk}\right)^{1/4}$, and the variables $\lambda = \frac{2m\alpha^2}{\hbar^2}E$ and $\rho = r/\alpha$, the Schroedinger’s equation becomes
\begin{align}
-\frac{d^2}{d\rho^2} u(\rho) + \rho^2 u(\rho) = \lambda u(\rho). \label{simpel SE}
\end{align}
See section \ref{opp d math} for futher details. 

Since we are working in radial coordinates we have $\rho \in [0,\infty)$. Since we can't represent infinity on a computer we have to find an aproximation, which we will come back to later. For now we define $\rho_{min}=10^{-6}$ and $\rho_{max}$ to represent the  minimum and maximum values of $\rho$. We used $10^{-6}$ instead of $0$ to avoid any potential divisions by zero.

Function \ref{simpel SE} is an differential equation that can be modeled similarly to earlier. If we have $N$ mesh points we get a step length

\begin{align}
h = \frac{\rho_{max} - \rho_{min}}{N}.
\end{align}
The value of $\rho$ at a point $i$ is then 
\begin{align}
\rho_i= \rho_0 + ih \hspace{1cm} i=1,2,\dots , N.
\end{align}
We can rewrite the Schroedinger equation for a value $\rho_i$ as
\begin{align}
-\frac{u_{i+1} -2u_i +u_{i-1}}{h^2}+\rho_i^2u_i = \lambda u_i).
\end{align}
We then introduced
\begin{align}
   d_i=\frac{2}{h^2}+\rho_{i}^2,
\end{align}
and
\begin{align}
e_{i} = -\frac{1}{\hbar^2},
\end{align}
which gives us
\begin{align}
d_iu_i+e_{i-1}u_{i-1}+e_{i+1}u_{i+1}  = \lambda u_i.
\end{align}
We then wrote the latter equation as a matrix eigenvalue problem
\begin{equation}
\begin{bmatrix}d_0 & e_0 & 0   & 0    & \dots  &0     & 0 \\
e_1 & d_1 & e_1 & 0    & \dots  &0     &0 \\
0   & e_2 & d_2 & e_2  &0       &\dots & 0\\
\dots  & \dots & \dots & \dots  &\dots      &\dots & \dots\\
0   & \dots & \dots & \dots  &\dots  e_{N-1}     &d_{N-1} & e_{N-1}\\
0   & \dots & \dots & \dots  &\dots       &e_{N} & d_{N}
\end{bmatrix}  \begin{bmatrix} u_{0} \\
u_{1} \\
\dots\\ \dots\\ \dots\\
u_{N}
\end{bmatrix}=\lambda \begin{bmatrix} u_{0} \\
u_{1} \\
\dots\\ \dots\\ \dots\\
u_{N}
\end{bmatrix}.  
\label{eq:sematrix}
\end{equation}

To solve the problem we implemented the matrix in Julia, and ran it through the algorithm in section \ref{rotator.jl}, with a tolerance of $10^{-4}$. The implemtation of the matrix can be seen in section \ref{opp_d matrix}. What we then did was test different approximations for $\rho_{max}$ and $N$ to find some values that are accurate and don't require a extemly large $N$. Our goal was to reproduce the analytical eigenvalues with four leading digits after the decimal point for the lowest eigenvalue. We looked both at the accuracy of the lowest eigenvalue, and used the Julia function $Statistics.std()$ on the difference between our numerical apppoximation and the analytical solutions, to get an idea of how close the results were overall. $std()$ is an function that finds the standar diviation for an array.


\subsection{Quantum dots in three dimensions, two electrons}
We then wanted to study two electrons in a harmonic oscillator well which also interact via a repulsive Coulomb interaction. The Schroedinger equation for two electrons with no repulsive Coulomb interaction is
\begin{align}
\left(  -\frac{\hbar^2}{2 m} \frac{d^2}{dr_1^2} -\frac{\hbar^2}{2 m} \frac{d^2}{dr_2^2}+ \frac{1}{2}k r_1^2+ \frac{1}{2}k r_2^2\right)u(r_1,r_2)  = E^{(2)} u(r_1,r_2) ,
\end{align}
and the repulsive Coulomb interaction between two electrons is
\begin{align}
V(r_1,r_2) = \frac{\beta e^2}{|\mathbf{r}_1-\mathbf{r}_2|}=\frac{\beta e^2}{r},
\end{align}
with $ \beta e^2 = 1.44 eVnm$. We introduce the relative coordinate $\mathbf{r} = \mathbf{r}_1-\mathbf{r}_2$, and the center-of-mass coordinate $\mathbf{R} = 1/2(\mathbf{r}_1+\mathbf{r}_2)$. The Schroedinger equation now reads
\begin{align}
\left(  -\frac{\hbar^2}{m} \frac{d^2}{dr^2} -\frac{\hbar^2}{4 m} \frac{d^2}{dR^2}+ \frac{1}{4} k r^2+  kR^2\right)u(r,R)  = E^{(2)} u(r,R).
\end{align}
The equations for $r$ and $R$ can be separated so that $u(r,R) = \psi(r)\phi(R)$ and the energy is given by the sumof the relative energy $E_r$ and the center-of-mass energy $E_R$, that is $ E^{(2)} = E_r + E_R$. After introducing the variables $\rho = r/\alpha$ and $\omega_r^2=\frac{1}{4}\frac{mk}{\hbar^2} \alpha^4$, the constants $\alpha = \frac{\hbar^2}{m\beta e^2}$ and $\lambda = \frac{m\alpha^2}{\hbar^2}E$, and only looking at $\psi (r)$ we are left with 
\begin{align}
  -\frac{d^2}{d\rho^2} \psi(\rho) + \omega_r^2\rho^2\psi(\rho) +\frac{1}{\rho} = \lambda \psi(\rho), \label{se 2 easy}
\end{align}
see section \ref{opp e math} for more details. Function \ref{se 2 easy} can be solved in almost the same way as for when we had one electron, only that the diagonal values $d_i$ now are $d_i = \frac{2}{h^2} + \omega_r^2\rho^2+1/\rho$. 

We now have an additional unknown value in $\omega_r$. We then studied the cases of $\omega_r = 0.01$, $\omega_r = 0.5$, $\omega_r = 1$, and $\omega_r = 5$, and use the values of $\rho_{max}$ and $N$ we found worked best for one electron.


\section{RESULTS}
\subsection{The implementation}
We see a fairly linear correlation between the number of matrix elements $n^2$ and the required amount similarity transformations. The computation time for each matrix was also proportional to $n^2$ In figure \ref{computation time plot} the tolerance for deviation from 0 in the non diagonal elements was $1e-4$. We found this to be the best balance between accuracy and efficiency. 
\begin{figure}[h!]
	\centering 
	%Scale angir størrelsen på bildet. Bildefilen må ligge i %samme mappe som tex-filen. 
	\includegraphics[scale=0.7]{../requiredRotations.pdf}
	\caption{A plot of the required number of rotations as a function of n: $rotations/n^2$, and the time used for calculating the eigenvalues of a $n\cdot n$ matrix.}
	%Label gjør det enkelt å referere til ulike bilder.
	\label{computation time plot}
\end{figure}

\subsection{Quantum dots in three dimensions, one electron}
We firstly tested with $\rho_{max}$ equal to all the whole numbers beteen 1 and 100 with $N=100$. We found that $\rho_{max} = 4$ had the closest lowest eigenvalue, with $2.9995$ to the analytical 3. This was within our desired four leading digits after the decimal point, however the standard deviation was quite high, as the error increased rapidly. You can see this in figure \ref{rhomaks 4}. The lowest std was for $\rho_{max} = 14$, but for it the lowest eigenvalue was only 2.99374, so we had to increase $N$. 

We then tested for values of $\rho_{max}$ between 2 and 25, with $N=200$. Again $\rho_{max} = 4$ had the closest lowest eigenvalue, but now $\rho_{max} = 15$ had the lowest std. Still, the lowest eigenvalue for $\rho_{max} = 15$ was only 2.99824.

We the increased $N$ to 400, which took a lot of processing time. The $\rho_{max}$ with the lowest std was now 23, but it had a lowest eigenvalue of 2.99897, which still wasn't close enough. 
We also ran the algorithm for $\rho_{max}=23$ and $N=500$, but it took a really long time, so we didn't run it for any other values. This time the lowest eigenvalue was 2.99934, which was within our desired limit. Due to the long calculation time we chose to use $\rho_{max}=23$ and $N=400$ when we calculated the interaction between two electrons. A plot of the eigenvalues for $\rho_{max}=23$ can be seen in figure \ref{rhomaks 23}, with both $N=400$ and $N=500$.

\begin{figure}[h!]
\centering 
%Scale angir størrelsen på bildet. Bildefilen må ligge i samme mappe som tex-filen. 
\includegraphics[scale=0.7]{../oppd_rho-4n-100.pdf}
\caption{A plot of the numerical solution against the analytical solution, when $\rho_{max}=4$ and $N=100$}
%Label gjør det enkelt å referere til ulike bilder.
\label{rhomaks 4}
\end{figure}

\begin{figure}[h!]
	\centering 
	%Scale angir størrelsen på bildet. Bildefilen må ligge i samme mappe som tex-filen. 
	\includegraphics[scale=0.7]{../oppd_rho-23n-400.pdf}
	\includegraphics[scale=0.7]{../oppd_rho-23n-500.pdf}
	\caption{A plot of the numerical solution against the analytical solution, when $\rho_{max}=23$, and $N=400$ and $N=500$}
	%Label gjør det enkelt å referere til ulike bilder.
	\label{rhomaks 23}
\end{figure}
	

\subsection{Quantum dots in three dimensions, two electrons}
The plots of the eigenvaules we found with diiferent $\omega_r$ can be seen in \ref{opp e res}.

\begin{figure}[h!]
\centering 
%Scale angir størrelsen på bildet. Bildefilen må ligge i samme mappe som tex-filen. 
\includegraphics[scale=0.45]{../oppe_rho-23n-400l0-0122946.pdf}
\includegraphics[scale=0.45]{../oppe_rho-23n-400l0-224796.pdf}
\includegraphics[scale=0.45]{../oppe_rho-23n-400l0-412688.pdf}
\includegraphics[scale=0.45]{../oppe_rho-23n-400l0-186125.pdf}
\caption{A plot of the numerical solution of numerical solution of two electrons in a harmonic oscillator well which also interact via a repulsive Coulomb interaction with $\omega_r$ equal to 0.01, 0.5, 1, and 5 respectively}
%Label gjør det enkelt å referere til ulike bilder.
\label{opp e res}
\end{figure}


\section{CONCLUSIONS}
With an accuracy of more digits than our editor cared to print out the Jacobi method, with a tolerance for non diagonal values of up to $1e-4$, seems to be an efficient and precise algorithm for computing eigenvectors and eigenvalues. The computation time is proportional to the number of matrix elements, and so the realistic limit for matrix size should be around $10^4 \cdot 10^4$. With $300 \cdot 300$ taking 12 seconds, $10^4 \cdot 10^4$ should take about
\begin{align}
\frac{(10^4)^2}{300^2}\cdot 12s\approx 4h
\end{align}
on a normal laptop. On a supercomputer this would of course be different, and i presume our implementation could have been further vectorised for greater efficiency.

When modeling the electrons we found that values for $\rho_{max}$ that had a low difference between the analytical and numerical solution for the lowest eigenvalue, often had a high standard deviation, and wise versa. From plot \ref{rhomaks 23} we do see that numerical and analytical solutions are very close, while the opposite is true for \ref{rhomaks 4}. Overall having a lower standard deviation is probally more important than having a small difference between the analytical and numerical solution for the lowest eigenvalue, but maybe a better compromise could have been achived. 

For the models of two electrons, we see that they vaguely follow a straigth line (or an s-curve maybye). If this is because of $\omega_r$ or something else is unknown.

\section{APENDICES}
\subsection{Integration loop from rotator.jl}\label{rotator.jl}
\lstinputlisting[language = python, firstline = 56, lastline = 131]{../rotator.jl}

\subsection{Test functions}\label{opp_c.jl}
\lstinputlisting[language = python, firstline = 3, lastline = 45]{../opp_c.jl}

\subsection{Math for Quantum dots in three dimensions, one electron}\label{opp d math}
We begin with 
\begin{align}
- \frac{\hbar^2}{2m} \left( \frac{1}{r^2} \frac{d}{dr} r^2 \frac{d}{dr} - \frac{l(l+1)}{r^2}\right) R(r) + V(r) R(r) = ER(r).
\end{align}
Firstly we had that $l=0$. We then substitued $R(r) = u(r)/r$, and got
\begin{align}
  -\frac{\hbar^2}{2 m} \frac{d^2}{dr^2} u(r) 
+ V(r)u(r)  = E u(r) .
\end{align}
In our case $V(r)$ is equal to $(1/2)kr^2$, where $k=m\omega^2$. We now introduce a dimensionless vaiable, $\rho = r/\alpha$, which means $V(\rho) = (1/2) k \alpha^2\rho^2$. This gives us
\begin{align}
-\frac{\hbar^2}{2 m \alpha^2} \frac{d^2}{d\rho^2} u(\rho) 
+ \frac{k}{2} \alpha^2\rho^2u(\rho)  &= E u(\rho) \\
 -\frac{d^2}{d\rho^2} u(\rho) 
+ \frac{mk}{\hbar^2} \alpha^4\rho^2u(\rho)  &= \frac{2m\alpha^2}{\hbar^2}E u(\rho) .
\end{align}
Finally we scale $\alpha$ so that $\frac{mk}{\hbar^2} \alpha^4 = 1$, and define $\lambda = \frac{2m\alpha^2}{\hbar^2}E$, which gives us
\begin{align}
  -\frac{d^2}{d\rho^2} u(\rho) + \rho^2u(\rho)  = \lambda u(\rho) .
\end{align}


\subsection{Implementing a matrix in Julia}\label{opp_d matrix}
\lstinputlisting[language = python, firstline = 5, lastline = 27]{../opp_d.jl}


\subsection{Math for Quantum dots in three dimensions, two electrons}\label{opp e math}
\begin{align}
\left(  -\frac{\hbar^2}{m} \frac{d^2}{dr^2} -\frac{\hbar^2}{4 m} \frac{d^2}{dR^2}+ \frac{1}{4} k r^2+  kR^2\right)u(r,R)  = E^{(2)} u(r,R).
\end{align}
Firstly we split $u(r,R)$ into  $\psi(r)$ and $\phi(R)$, with $E^{(2)} = E_r + E_R$. We are only interested in $E_r$, so we can remove everything related to $R$. We then add the repulsive Coulomb interaction $V(r_1,r_2)$, giving us
\begin{align}
\left(  -\frac{\hbar^2}{m} \frac{d^2}{dr^2}+ \frac{1}{4}k r^2+\frac{\beta e^2}{r}\right)\psi(r)  = E_r \psi(r).
\end{align}
We then introduced the dimensionless vaiable $\rho = r/\alpha$, giving us
\begin{align}
  -\frac{d^2}{d\rho^2} \psi(\rho) 
+ \frac{1}{4}\frac{mk}{\hbar^2} \alpha^4\rho^2\psi(\rho)+\frac{m\alpha \beta e^2}{\rho\hbar^2}\psi(\rho)  = 
\frac{m\alpha^2}{\hbar^2}E_r \psi(\rho) .
\end{align}
We then define $\omega_r^2=\frac{1}{4}\frac{mk}{\hbar^2} \alpha^4$, $\frac{m\alpha \beta e^2}{\hbar^2}=1$ and $\lambda = \frac{m\alpha^2}{\hbar^2}E$, giving us
\begin{align}
  -\frac{d^2}{d\rho^2} \psi(\rho) + \omega_r^2\rho^2\psi(\rho) +\frac{1}{\rho} = \lambda \psi(\rho).
\end{align}


\section{REFERENCES}
\begin{thebibliography}{9}
	\bibitem{lecture notes}
	Computational Physics, Lecture Notes Fall 2015, Morten Hjort-Jensen p.215-220
\end{thebibliography}




%\begin{figure}[h!]
%	\centering 
%	%Scale angir størrelsen på bildet. Bildefilen må ligge i samme mappe som tex-filen. 
%	\includegraphics[scale=0.7]{opp2_7.pdf}
%	\caption{A plot of the entropy}
%	%Label gjør det enkelt å referere til ulike bilder.
%	\label{2.7}
%\end{figure}






















\end{document}