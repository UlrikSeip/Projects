\documentclass[a4paper]{article}
% Import some useful packages
\usepackage[margin=0.5in]{geometry} % narrow margins
\usepackage[utf8]{inputenc}
\usepackage[english]{babel}
\usepackage{hyperref}
\usepackage{listings}
\usepackage{amsmath,graphicx,varioref,verbatim,amsfonts,geometry,amssymb,dsfont,blindtext}
%\usepackage{minted}
\usepackage{amsmath}
\usepackage{xcolor}
\hypersetup{colorlinks=true}
\definecolor{LightGray}{gray}{0.95}
\definecolor{dkgreen}{rgb}{0,0.6,0}
\definecolor{gray}{rgb}{0.5,0.5,0.5}
\definecolor{mauve}{rgb}{0.58,0,0.82}
\definecolor{mygray}{rgb}{0.9,0.9,0.9}
\definecolor{LightGray}{gray}{0.95}
\lstset{frame=tb,
	language=Python,
	aboveskip=3mm,
	belowskip=3mm,
	showstringspaces=false,
	columns=flexible,
	basicstyle={\small\ttfamily},
	numbers=none,
	numberstyle=\tiny\color{gray},
	keywordstyle=\color{blue},
	commentstyle=\color{dkgreen},
	stringstyle=\color{mauve},
	backgroundcolor=\color{mygray}
	%breaklines=true,
	%breakatwhitespace=true,
	%tabsize=3
}
\title{Project 2 in FYS3150}
\author{Bendik Steinsvåg Dalen, Ulrik Seip}
%\renewcommand\thesection.\alph{section}
%\renewcommand\thesection{\Alph{section}}
\renewcommand\thesubsection{\thesection.\alph{subsection}}
\renewcommand\thesubsubsection{\thesubsection.\roman{subsubsection}}
\begin{document}
\maketitle

\section{ABSTRACT}
In this project we have coded a implementation of Jacobi’s rotation algorithm, that finds an approximation to differential equations. We then then used this to model a harmonic oscillator problem in three dimensions, with one and two electrons. 

\section{INTRODUCTION}


\section{METHOD}

\subsection{Basic code, eller noe}

\subsection{Testing the code}

\subsection{Quantum dots in three dimensions, one electron}
Now that we had a general algorithm we used it to model a electron that moves in a three-dimensional harmonic oscillator potential. In other words, we looked for the solution of the radial part of Schroedinger’s
equation for one electron, which reads
\begin{align}
- \frac{\hbar^2}{2m} \left( \frac{1}{r^2} \frac{d}{dr} r^2 \frac{d}{dr} - \frac{l(l+1)}{r^2}\right) R(r) + V(r) R(r) = ER(r).
\end{align}
This problem also has analytical solutions, so we can test how accurate our algorithm is.

(Some math-stuff).

The Schroedinger’s equation then becomes
\begin{align}
-\frac{d^2}{d\rho^2} u(\rho) + \rho^2 u(\rho) = \lambda y(\rho). \label{simpel SE}
\end{align}
Since we are working in radial coordinates we have $\rho \in [0,\infty)$. Since we can't represent infinity on a computer we have to find an aproximation, which we will come back to later. For now we define $\rho_{min}=0$ and $\rho_{max}$ to represent the  minimum and maximum values of $\rho$. 

Function \ref{simpel SE} is an differential equation that can be modeled similarly to (ting). If we have $n$ mesh points we get a step length
\begin{align}
h = \frac{\rho_{max} - \rho_{min}}{n}.
\end{align}


\subsection{Quantum dots in three dimensions, two electrons}


\section{RESULTS}


\section{CONCLUSIONS}

\section{APENDICES}

\section{REFERENCES}





%\begin{figure}[h!]
%	\centering 
%	%Scale angir størrelsen på bildet. Bildefilen må ligge i samme mappe som tex-filen. 
%	\includegraphics[scale=0.7]{opp2_7.pdf}
%	\caption{A plot of the entropy}
%	%Label gjør det enkelt å referere til ulike bilder.
%	\label{2.7}
%\end{figure}






















\end{document}